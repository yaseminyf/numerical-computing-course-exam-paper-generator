Si consideri la funzione $f(x)=x \sin(x)-x/2$.
\begin{itemize}
\item  Localizzare le radici e
scrivere gli intervalli in cui si trovano.
\item Verificare se nell' intervallo $[2,4]$ esiste una radice e se valgono le
condizioni di convergenza del metodo di Newton.
\item Applicare comunque nell' intervallo $[2,4]$ un passo del metodo di Newton alla funzione $f$
usando un calcolatore con base $\beta =10$, $m=3$ cifre di
mantissa, nessuna limitazione sulla caratteristica e tecnica di
arrotondamento, per approssimare ciascuna radice. (Eseguire tutti
i calcoli in aritmetica finita)
\end{itemize}
