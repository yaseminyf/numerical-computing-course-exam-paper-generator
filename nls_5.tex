Si consideri la funzione $f(x)=x^2-4x-5$. Localizzare le radici e 
scrivere gli intervalli in cui si trovano.
\medskip
\[
\left [
\begin{array}{cccccccccccccc}
\quad &  \quad & \quad &  \quad & \quad & \quad & \quad & \quad  \\
\quad &  \quad & \quad &  \quad & \quad &  \quad & \quad & \quad \\
\quad &  \quad & \quad &  \quad  & \quad  & \quad & \quad & \quad
\end{array}\right]
\]
\medskip
\[
\left [
\begin{array}{cccccccccccccc}
\quad &  \quad & \quad &  \quad & \quad & \quad & \quad & \quad  \\
\quad &  \quad & \quad &  \quad & \quad &  \quad & \quad & \quad \\
\quad &  \quad & \quad &  \quad  & \quad  & \quad & \quad & \quad
\end{array}\right]
\]
\noindent Dire a quale approssimazione della soluzione di $f(x)=0$ 
si perviene applicando alla funzione $f$ un passo del metodo delle 
secanti a partire dall'intervallo $[-2,1]$.
\medskip
\[
\left [
\begin{array}{cccccccccccccc}
\quad &  \quad & \quad &  \quad & \quad & \quad & \quad & \quad  \\
\quad &  \quad & \quad &  \quad & \quad &  \quad & \quad & \quad \\
\quad &  \quad & \quad &  \quad  & \quad  & \quad & \quad & \quad
\end{array}\right]
\]
