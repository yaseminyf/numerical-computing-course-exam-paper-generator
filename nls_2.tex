Si consideri la funzione $f(x)=x^2-4x-5$. Localizzare le radici 
e scrivere gli intervalli in cui si trovano. 
\bigskip
\[
\left [
\begin{array}{cccccccccccccc}
\quad &  \quad & \quad &  \quad & \quad & \quad & \quad & \quad  \\
\quad &  \quad & \quad &  \quad & \quad &  \quad & \quad & \quad \\
\quad &  \quad & \quad &  \quad  & \quad  & \quad & \quad & \quad
\end{array}\right]
\]
\bigskip
\[
\left [
\begin{array}{cccccccccccccc}
\quad &  \quad & \quad &  \quad & \quad & \quad & \quad & \quad  \\
\quad &  \quad & \quad &  \quad & \quad &  \quad & \quad & \quad \\
\quad &  \quad & \quad &  \quad  & \quad  & \quad & \quad & \quad
\end{array}\right]
\]
\bigskip

\noindent A partire dall'intervallo $[-2,1]$ dire a quale 
intervallo si arriva dopo aver implementato $2$ passi del metodo di bisezione.
\bigskip
\[
\left [
\begin{array}{cccccccccccccc}
\quad &  \quad & \quad &  \quad & \quad & \quad & \quad & \quad  \\
\quad &  \quad & \quad &  \quad & \quad &  \quad & \quad & \quad \\
\quad &  \quad & \quad &  \quad  & \quad  & \quad & \quad & \quad
\end{array}\right]
\]
