Consideriamo un elaboratore operante con rappresentazione in base 
$\beta=10$, aritmetica floating-point e tecnica di 
arrotondamento. Siano $m=4$ le cifre a disposizione
della mantissa e $n=2$ le cifre per la caratteristica.\\
\noindent Riportando lo svolgimento, si determini la 
rappresentazione floating-point $\widetilde{x}$ e $\widetilde{y}$ 
dei seguenti vettori:
\[ x=(6.2177,\,  0.0031299), \quad \quad  y=(27.566, \, 10^{102}) \]
\vspace*{5cm}
\[ 
\widetilde{x}=\left [
\begin{array}{cccccccccccccc}
\quad &  \quad & \quad &  \quad & \quad & \quad & \quad & \quad  & \quad  & \quad & \quad     \\
\quad &  \quad & \quad &  \quad  & \quad  & \quad & \quad & \quad 
& \quad & \quad & \quad
\end{array}\right]
\]
\medskip
\[
\widetilde{y}=\left [
\begin{array}{cccccccccccccc}
\quad &  \quad & \quad &  \quad & \quad & \quad & \quad & \quad & \quad  & \quad  & \quad   \\
\quad &  \quad & \quad &  \quad  & \quad  & \quad & \quad & \quad 
& \quad & \quad & \quad
\end{array}\right]
\]
\noindent
Usando l'aritmetica dell'elaboratore si calcoli $\|x\|_1$ 
riportando lo svolgimento.
\vspace*{3cm}
\[ 
\|x\|_1=\left [
\begin{array}{ccc}
\quad &  \quad & \quad \\
\quad &  \quad & \quad
\end{array}\right]
\]
