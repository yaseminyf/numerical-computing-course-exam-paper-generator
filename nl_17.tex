Si consideri la funzione $f(x)=x \sin(x)-x/2$.
\begin{itemize}
\item  Localizzare le radici e
scrivere intervalli in cui si trovano.
\item Verificare che esiste una radice nell' intervallo $[0.51, 0.62]$ e comunque applicare un passo del
metodo di bisezione alla funzione $f$ usando un calcolatore con
base $\beta =10$, $m=3$ cifre di mantissa, nessuna limitazione
sulla caratteristica e tecnica di arrotondamento, per approssimare
ciascuna radice. (Eseguire tutti i calcoli in aritmetica finita)
\end{itemize}
