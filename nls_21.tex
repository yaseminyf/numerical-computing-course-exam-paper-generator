Si consideri la funzione $f(x)=x^2-5x+4$.
\begin{itemize}
\item  Localizzare le radici e
scrivere gli intervalli in cui si trovano.
\item Verificare che nell' intervallo $[3,5]$ esiste una radice e comunque applicare due passi del
metodo di Bisezione alla funzione $f$ usando un calcolatore con
base $\beta =10$, $m=3$ cifre di mantissa, nessuna limitazione
sulla caratteristica e tecnica di arrotondamento. (Eseguire tutti
i calcoli in aritmetica finita)
\end{itemize}
