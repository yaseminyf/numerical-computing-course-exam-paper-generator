\documentclass[11pt]{article}
\hoffset=-15mm \voffset=-15mm
\textheight=225mm \textwidth=160mm
%\evensidemargin=10mm
\oddsidemargin=18mm

\def\N{\hbox{I\kern-.2em\hbox{N}}}
\def\P{\hbox{I\kern-.2em\hbox{P}}}
\def\R{\hbox{\rm I\kern-.2em\hbox{\rm R}}}
\def\Z{\hbox{\rm Z\kern-4.0pt{\rm Z}}}
\def\conv{{\hbox{$ \bigcirc \kern-0.6em \scriptstyle  5  $}}}
\def\ee{\`e }
%%  Matrices
\newcommand{\svec}[2]{\begin{pmatrix} #1 \\ #2 \end{pmatrix}}
\newcommand{\smat}[4]{\begin{pmatrix} #1 & #2 \\ #3 & #4 \end{pmatrix}}
\newcommand{\narrowbrace}[2]{\,\,\underbrace{\!\!#1\!\!}_{#2}\,\,}

%%  Latin Abbreviations
%\newcommand\eg{{\it\thinspace e.g.}}
\newcommand\ie{{\it\thinspace i.e.}}
\begin{document}
\vskip 10pt \noindent


\vskip 20pt
\noindent

 \vskip 20 pt \noindent 
\vskip 20pt
\noindent
\vskip 20pt  \noindent {\bf Esercizio 1.}
Consideriamo un elaboratore operante con
rappresentazione in base $\beta=10$, aritmetica floating-point e
tecnica di troncamento. Siano  $m=4$ le cifre a disposizione della mantissa
e  $n=1$ le cifre per la  caratteristica, escluso il segno.
\vskip 2pt
\noindent
i) Si indichi l'insieme ${\cal F}$ dei numeri di macchina ed il valore della precisione di macchina.
\vskip 2pt    \noindent
ii) Si  rappresentino in macchina   le seguenti matrici
$$
A=
\left(
\begin{array}{rr}
-56.924 \,& -4.008765\\
-0.00452 \,&  1.1
\end{array}
\right)
\quad \quad
B=
\left(
\begin{array}{rr}
-56.0\, 10^7\, & -40.9\, 10^8\\
0.1\, 10^{-12} \,&  0.0
\end{array}
\right)
\quad \quad
$$
\vskip 2pt    \noindent
iii)
Usando l'aritmetica dell'elaboratore  si calcoli la norma 1 e la norma infinito di $A$.
\medskip \noindent {\bf Svolgimento Esercizio 1.}
\\
i)   L'insieme  ${\cal F}$ dei numeri di macchina  \`e  costituito dai numeri
$x\in \R$ della forma
$$
x=\pm \, 0.a_1a_2a_3a_4\, \beta^b, \quad \quad \mbox{dove }\quad \quad  a_1\neq 0, \quad \beta=10, \quad
 -9\le b\le 9.
$$
Quindi si indica ${\cal F}(\beta, m, L, U)={\cal F}(10, 4, -9, 9)$.
\\
Dato che l'elaboratore opera per troncamento, la precisione di macchina
\`e $\epsilon_m=\beta^{1-m}=10^{-3}$.
\vskip 4pt
\noindent
ii) Rappresentiamo in macchina   le matrici $A$ e $B$:
\begin{eqnarray}
A&=&
\left(
\begin{array}{ll}
-0.56924\, 10^2 \,& -0.4008765\, 10^1\\
-0.452\, 10^{-2}  \,&  +0.11  \, 10^1
\end{array}
\right)
\quad \quad  \Rightarrow  \quad fl(A)=\left(
\begin{array}{ll}
-0.5692\, 10^2 \,& -0.4008\,\, 10^1\\
-0.4520\, \,10^{-2}  \,&  +0.1100 \, \, 10^1
\end{array}
\right)    \nonumber \\
& & \nonumber \\
& & \nonumber \\
B&=&
\left(
\begin{array}{ll}
-0.560\, 10^9\,\, & -0.409\,\, 10^{10} \\
+0.1\,\, 10^{-12}  &  +0.0
\end{array}
\right)  \quad \quad   \Rightarrow \quad fl(B)= \left(
\begin{array}{ll}
-0.5600\, 10^9\, & \overbrace{-0.4090\, 10^{10}}^{overflow} \\
\underbrace{+0.1000\, 10^{-12}  }_{underflow}\,&  +0.0000\,\,10^0
\end{array}
\right)   \nonumber
\end{eqnarray}
\vskip 10pt
\noindent
iii) Posto $A=(a_{ij})\in \R^{2\times 2}$  abbiamo
$\|fl(A)\|_1=\max\left\{|a_{11}|+|a_{21}|,\, |a_{12}|+|a_{22}| \right\} $
e $\|fl(A)\|_{\infty}=\max\left\{|a_{11}|+|a_{12}|,\, |a_{21}|+|a_{22}| \right\}$.
\\
Usando l'aritmetica dell'elaboratore calcoliamo $\|fl(A)\|_1$ e $\|fl(A)\|_{\infty}$.
\begin{eqnarray}
\|fl(A)\|_1&=&\max \left\{0.5692\, 10^2 \oplus 0.4520\, \,10^{-2},\, 0.4008\, 10^1 \oplus 0.1100 \, \, 10^1   \right\}
\\ \nonumber
& =& \max\left\{0.5692\, 10^2 \oplus 0.0000\, \,10^{2},\,0.4008\, 10^1 \oplus 0.1100 \, \, 10^1   \right\} =0.5692\, 10^2
\end{eqnarray}
\begin{eqnarray}
\|fl(A)\|_\infty&=&\max \left\{0.5692\, 10^2 \oplus 0.4008 \, 10^1 \, 0.4520\, \,10^{-2}  \oplus 0.1100 \, \, 10^1   \right\}
\\ \nonumber
& =& \max\left\{0.5692\, 10^2 \oplus 0.0400\, \,10^{2},\,0.0004\, 10^1 \oplus 0.1100 \, \, 10^1   \right\} =0.6092\, 10^2
\end{eqnarray}
\vskip 20pt  \noindent {\bf Esercizio 1.}
Consideriamo un elaboratore operante con
rappresentazione in base $\beta=10$, aritmetica floating-point e
tecnica di troncamento. Siano  $m=4$ le cifre a disposizione della mantissa
e  $n=1$ le cifre per la  caratteristica, escluso il segno.
\\
i) Si indichi l'insieme ${\cal F}$ dei numeri di macchina ed il valore della precisione di macchina.
\\
ii) Si  rappresentino in macchina   le seguenti matrici
$$
A=
\left(
\begin{array}{rr}
-68.7254 \, & -7.008765\\
-2.452 \,&  0.08765
\end{array}
\right)
\quad \quad
B=
\left(
\begin{array}{rr}
-56.0\, 10^7 \,& -40.9\, 10^8\\
0.1\, 10^{-12} \,&  0.0
\end{array}
\right)
\quad \quad
$$
\\
iii)
Usando l'aritmetica dell'elaboratore  si calcoli la norma 1 e la 
\medskip \noindent {\bf Svolgimento Esercizio 1.}
\\
\underline{Si procede come nell'Esercizio 1, pagina 14.}
La rappresentazione floating point di $A$ e $B$ \ee
\begin{eqnarray}
fl(A)&=&
\left(
\begin{array}{ll}
-0.6872\, 10^2 \,& -0.7008\, 10^1\\
-0.2452\, 10^{1}  \,&  +0.8765 \, 10^{-1}
\end{array}
\right)
\quad \quad
fl(B)= \left(
\begin{array}{ll}
-0.5600\, 10^9\, & \overbrace{-0.4090\, 10^{10}}^{overflow} \\
\underbrace{+0.1000\, 10^{-12}  }_{underflow}\,&  +0.0000\,\,10^0
\end{array}
\right)   \nonumber
\end{eqnarray}
\vskip 10pt \noindent
Inoltre $\|fl(A)\|_1=0.7117\, 10^2$,  $\,\,\|fl(A)\|_\infty =0.7572\, 10^2$.norma infinito di $A$.
\medskip \noindent 
\medskip  \medskip\medskip\noindent

\vskip 20pt
\noindent

\vskip 30pt
\noindent

\end{document}