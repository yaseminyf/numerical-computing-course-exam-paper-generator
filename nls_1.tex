Si consideri la funzione $f(x)=x^2-4x-5$ nell'intervallo 
$[-2,1]$. Dire quanti passi del metodo di bisezione occorrono per 
approssimare la radice $-1$ con un errore assoluto inferiore di 
$1/5$.
\medskip
\[
\left [
\begin{array}{cccccccccccccc}
\quad &  \quad & \quad &  \quad & \quad & \quad & \quad & \quad  \\
\quad &  \quad & \quad &  \quad & \quad &  \quad & \quad & \quad \\
\quad &  \quad & \quad &  \quad  & \quad  & \quad & \quad & \quad
\end{array}\right]
\]

\medskip
\noindent Dire a quale approssimazione si arriva dopo aver 
implementato $2$ passi del metodo di bisezione.\\
\medskip
\[
\left [
\begin{array}{cccccccccccccc}
\quad &  \quad & \quad &  \quad & \quad & \quad & \quad & \quad  \\
\quad &  \quad & \quad &  \quad & \quad &  \quad & \quad & \quad \\
\quad &  \quad & \quad &  \quad  & \quad  & \quad & \quad & \quad
\end{array}\right]
\]
