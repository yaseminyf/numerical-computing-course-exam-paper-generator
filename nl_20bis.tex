Si consideri la funzione $f(x)=x \sin(x)-x/2$.
\begin{itemize}
\item Verificare che nell' intervallo $[-4, -1]$ esiste una radice di $f(x)$
\item Applicare $2$ passi del metodo di Newton alla funzione $f(x)$ per approssimare la radice nell' intervallo di cui al punto precedente a partire da un punto opportuno
usando un calcolatore con base $\beta =10$, $m=3$ cifre di
mantissa, nessuna limitazione sulla caratteristica e tecnica di
arrotondamento.
\end{itemize}
