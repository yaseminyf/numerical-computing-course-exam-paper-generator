\documentclass[11pt]{article}
\hoffset=-15mm \voffset=-15mm \textheight=225mm \textwidth=160mm
%\evensidemargin=10mm
\oddsidemargin=18mm
%\pagestyle{empty}
\newtheorem{teo}{Theorem}[section]
\newtheorem{defi}{Definition}[section]
\newtheorem{lem}{Lemma}[section]
\newtheorem{cor}{Corollary}[section]
\newcommand{\ha}[1]{\hat{#1}}
\newcommand{\nw}{Newton's }
\newcommand{\nlm}{Newton-like methods }
\newcommand{\nlmm}{Newton-like methods}
\newcommand{\nin}{Inexact Newton  }
\newcommand{\ninl}{Inexact Newton-like methods  }
\newcommand{\ninlm}{Inexact Newton-like methods}
\newcommand{\eb}{{\bf \tau_\nu }}
\def\IR{\hbox{\rm I\kern-.2em\hbox{\rm R}}}
\def\IC{\hbox{\rm C\kern-.58em{\raise.53ex\hbox{$\scriptscriptstyle|$}}
    \kern-.55em{\raise.53ex\hbox{$\scriptscriptstyle|$}} }}
\begin {document}
%%%%%%%%%%%%%%%%%%%%%%%%%%%%%%%%%
\noindent {\bf Esercizio n.~1} Consideriamo un elaboratore operante con rappresentazione in base
$\beta=10$, aritmetica floating-point e tecnica di
arrotondamento. Siano  $m=4$ le cifre a disposizione della
mantissa e $n=2$ le cifre per la  caratteristica.

\begin{itemize}
 \item Si scriva
quant'\`{e} la precisione di macchina $\varepsilon_m$
dell'elaboratore sopra descritto.
\item
Si determini la rappresentazione floating-point $\widetilde{x}$ e
$\widetilde{y}$ dei seguenti vettori:
\[ x=(6.2177,\,  0.0031299, -5.6), \quad \quad  y=(27.566, \, 10^{102},\,  0.1\,\,10^{99}) \]
\end{itemize}


\noindent {\bf Esercizio n.~2} Consideriamo un elaboratore operante con rappresentazione in base 
$\beta=10$, aritmetica floating-point e tecnica di 
arrotondamento. Siano $m=4$ le cifre a disposizione
della mantissa e $n=2$ le cifre per la caratteristica.\\

\noindent Si determini la rappresentazione floating-point 
$\widetilde{x}$ e $\widetilde{y}$ dei seguenti vettori:
\[ x=(6.2177,\,  0.0031299), \quad \quad  y=(27.566, \, 0.1^{2}) \]

\noindent Usando l'aritmetica dell'elaboratore si calcolino $\|x\|_1$.


\noindent {\bf Esercizio n.~3} Consideriamo un elaboratore operante con rappresentazione in base 
$\beta=10$, aritmetica floating-point e tecnica di 
arrotondamento. Siano $m=3$ le cifre a disposizione della 
mantissa e $n=2$ le cifre per la caratteristica. Si scriva 
quant'\`{e} la precisione di macchina $\varepsilon_m$ 
dell'elaboratore sopra descritto.\\

\noindent Si determini la rappresentazione floating-point 
$\widetilde{A}$ e $\widetilde{B}$ delle seguenti matrici:
\[
A=\left(
\begin{array}{ccc}
0.0034 && -1.756 \\
-10.42 && 5.5
\end{array}
\right ) \quad \quad B=\left(
\begin{array}{ccc}
17.8\, 10^{9} & & 9.0 \\
0.0001 &  & 10^{12}
\end{array}
\right )
\]



\noindent {\bf Esercizio n.~4} Consideriamo un elaboratore operante con rappresentazione in base
$\beta=10$, aritmetica floating-point e tecnica di
arrotondamento. Siano $m=3$ le cifre a disposizione
della mantissa e $n=1$ le cifre per la caratteristica.\\
\begin{itemize}
\item
\noindent Si determini la rappresentazione floating-point
$\widetilde{A}$ della seguente matrice:
\[
A=\left  (
\begin{array}{ccc}
0.0034 && -1.756 \\
-10.42 && 5.5
\end{array}
\right )
\]

\item
\noindent Usando l'aritmetica dell'elaboratore si calcoli
$\|A\|_\infty$.

\end{itemize}


\noindent {\bf Esercizio n.~5} Consideriamo un elaboratore operante con rappresentazione in base 
$\beta=10$, aritmetica floating-point e tecnica di 
arrotondamento. Siano $m=4$ le cifre a disposizione
della mantissa e $n=2$ le cifre per la caratteristica.\\

\noindent Riportando lo svolgimento, si determini la 
rappresentazione floating-point $\widetilde{x}$ e $\widetilde{y}$ 
dei seguenti vettori:
\[ x=(6.2177, \,  0.0031299), \quad \quad  y=(27.566, \, 10^{102}) \]

\noindent
Usando l'aritmetica dell'elaboratore si calcoli $\|x\|_1$ 
riportando lo svolgimento.



\noindent {\bf Esercizio n.~6} Consideriamo un elaboratore operante con rappresentazione in base
$\beta=10$, aritmetica floating-point e tecnica di arrotondamento.
Siano $m=3$ le cifre a disposizione della mantissa e $n=1$ le
cifre per la caratteristica.
\begin{itemize}
\item Si scriva quant'\`{e} la precisione di macchina
$\varepsilon_m$ dell'elaboratore sopra descritto.
\item Si
determini la rappresentazione floating-point $\widetilde{A}$ e
$\widetilde{B}$ delle seguenti matrici:
\[
A=\left(
\begin{array}{ccc}
0.0034 && -1.756 \\
-10.42 && 5.5
\end{array}
\right) \quad \quad B=\left(
\begin{array}{ccc}
17.8\, 10^{9} & & 9.0 \\
0.0001 &  & 10^{-12}
\end{array}
\right)
\]
\end{itemize}


\noindent {\bf Esercizio n.~7} Consideriamo un elaboratore operante con rappresentazione in base
$\beta=10$, aritmetica floating-point e tecnica di arrotondamento.
Siano  $m=3$ le cifre a disposizione della mantissa e $n=2$ le
cifre per la  caratteristica.

\begin{itemize}
\item Si scriva
quant'\`{e} la precisione di macchina $\varepsilon_m$
dell'elaboratore sopra descritto.

\item Si determini la rappresentazione floating-point $\widetilde{x}$ e
$\widetilde{y}$ dei seguenti vettori:
\[ x=(3.123,\,  0.00319, 6.6), \quad \quad  y=(17.566, \, 10^{201},\,  0.01\,\,10^{99}) \]
\item Si dica quali sono i valori esattamente rappresentabili in
macchina.
\end{itemize}


\noindent {\bf Esercizio n.~8} Consideriamo un elaboratore operante con rappresentazione in base
$\beta=10$, aritmetica floating-point e tecnica di
arrotondamento. Siano $m=4$ le cifre a disposizione
della mantissa e $n=2$ le cifre per la caratteristica.\\
\begin{itemize}
\item
\noindent Si determini la rappresentazione floating-point
$\widetilde{x}$ e $\widetilde{y}$ dei seguenti vettori:
\[ x=(3.2177,\,  0.031299), \quad \quad  y=(31.677, \, 0.1^{2}) \]

\item
\noindent Usando l'aritmetica dell'elaboratore si calcolino $\|x\|_1$.
\end{itemize}



\end{document}
